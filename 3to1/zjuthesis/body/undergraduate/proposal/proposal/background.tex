\section{问题提出的背景}

\subsection{背景介绍}
半导体是一种电导率在绝缘体至导体之间的物质或材料。半导体器件利用此特性在电子技术、清洁能源、制造业和自动化等许多领域具有重要意义。半导体宏观数学模型则是现代半导体工业的重要研究课题之一,它不仅为半导体材料、微电子配件等相关领域的许多技术问题提供了理论上必要的解释,提供对器件行为和性能的预测和分析,利于半导体材料开发和优化。
\subsubsection{DD模型\cite{cercignani2000device}和HF模型\cite{cercignani2000device}}
漂移-扩散(Drift-Diffusion,简称DD)模型是一种相对简单且广泛使用的半导体材料宏观数学模型。
该模型用于描述半导体器件中的载流子输运行为,特别是针对低频和直流条件下的行为。
高场模型(High-field model,简称HF)是半导体宏观数学模型的一种,用于描述在高电场条件下的半导体器件行为。
二者来源于描述半导体器件中电子输运的经典Boltzmann-Poisson系统。
这两个模型都存在一阶导数对流项和二阶导数扩散(热传导)项,并且对流-扩散系统与Poisson电势方程耦合\cite{cercignani2000device}。
分析中的主要技术难点是处理电子势和电子浓度之间的非线性耦合。
此外,在半导体器件中存在尖锐的不连续掺杂剖面,这对数值模拟构成了另一个挑战。
\subsubsection{discontinuous Galerkin (DG)方法}
1973年,Reed和Hill首次提出DG法\cite{reed1973triangular}。
Cockburn等发表了一系列论文极大发展了DG法,建立了DG法解决非线性时间依赖问题的基本框架\cite{reed1973triangular,cockburn1991runge,cockburn1989tvb2,cockburn1989tvb3,cockburn1990runge,cockburn1998runge},比如空气动力学的欧拉方程,使用显式非线性稳定的高阶Runge-Kutta(RK)时间离散和DG空间离散化方法,通过精确或近似的Riemann求解器作为界面通量以及total variation bounded(TVB)非线性限制器,取得了强激波下的非振荡性质。

DG方法在诸多领域中得到了迅速应用,包括声学、电磁学、气体动力学、颗粒流动、磁流体力学、气象学、浅水模拟、海洋学、油藏模拟、半导体器件模拟、多孔介质中的污染物传输、涡轮机械、湍流流动、粘弹性流体和天气预报等\cite{cockburn2000development}。

\subsubsection{local discontinuous Galerkin (LDG)方法}
对于包含高阶空间导数的方程,如对流-扩散方程
\begin{equation}\label{eq:convection-diffusion}
    u_t + \sum f_i(u)_{x_i} - \sum \sum (a_{i,j}(u)u_{x_j})_{x_i} = 0,
\end{equation}
DG方法不能直接应用。这是因为解空间由在单元交界面上不连续的分段多项式组成,不足以处理高阶导数。这是有限元中的典型“不相容”情况。将DG方法粗心地直接应用于包含二阶导数的热方程可能会得到计算中表现良好但与原方程“不一致”的方法,并且对精确解存在O(1)的误差\cite{cockburn2001runge,zhang2003analysis}。

对于包含高阶导数的时间相关偏微分方程,如对流-扩散方程\eqref{eq:convection-diffusion},LDG方法的思想是将方程重写为一阶系统,然后在该系统上应用DG方法。这个方法成功的关键因素是正确设计界面数值通量。这些通量必须保证稳定性和所有引入的辅助变量的局部可解性,这些辅助变量用于逼近解的导数。所有辅助变量局部可解解释该方法被称为“局部”不连续Galerkin方法的原因\cite{cockburn1998local}。
1998年,Cockburn和Shu首次提出了解决时间依赖对流-扩散系统\autoref{eq:convection-diffusion}的LDG法\cite{cockburn1998local}。他们的工作受到了Bassi和Rebay在可压缩Navier-Stokes方程中成功的数值实验的启发\cite{bassi1997high}。

2010年,Xu等提出解决高阶时间依赖PDE的LDG方法\cite{xu2010local},并证明了其好的性质。它们可以轻松地设计为任何精度。实际上,精度的阶数可以在每个单元中进行局部确定,这允许高效的p自适应性。它们可以用于任意三角剖分,甚至包括具有悬点(不同单元共享的节点,但具有不同属性)的三角剖分,这样可以实现高效的h自适应性。这些方法具有出色的并行效率,因为它们在本质上是非常局部的,每个单元只需要与其直接相邻的单元进行通信,而不受精度的影响。此外,这些方法具有优秀的可证明非线性稳定性。

\subsubsection{项目提出的原因}
我们已经开发了一种局部不连续Galerkin(LDG)有限元方法来解决半导体器件仿真中的时间依赖和稳态矩模型问题\cite{liu2004local,liu2007locala}。这种模型中同时存在一阶导数对流项和二阶数扩散(热传导)项,并且对流-扩散系统与泊松电势方程耦合。对流-扩散系统由局部不连续Galerkin(LDG)方法进行离散化。电场的电势方程也通过LDG方法进行离散化处理。通过使用LDG方法对一阶和高阶空间导数进行统一离散化,可完全发挥该方法在高度h-p自适应性和并行效率方面的潜力。
该方法具有良好的分辨度\cite{liu2004local,liu2007locala},并与ENO有限差分方法\cite{jerome1994energy}得到的结果达成一致。

我们希望延续之前的工作\cite{liu2004local,liu2007locala},对具有平滑解的半导体器件的一维漂移-扩散(DD)模型和高场(HF)模型进行误差分析,二者都是使用广泛的半导体器材模型。我们注意到,在文献[12, 13]中,分别对解DD模型的$P^1$连续有限元方法以及与$P^0-P^1$混合有限元方法耦合求解泊松方程的方法进行了分析。
对于我们的情况,分析中的主要技术困难包括处理跨单元跳跃项,这些项在通过泊松求解器与非线性耦合时产生于我们的数值方法的不连续性。
\subsection{本研究的意义和目的}
在之前的研究Liu-Shu中已经证明LDG法对于部分半导体器材模型能够取得好的结果,因此我们希望探究LDG法能否在DD模型和HF模型上也取得好的结果。由于LDG法具有高阶精度和高分辨率的特点,可以提供更准确的数值结果。并且LDG法具有高度的h-p自适应性,适用于各种不规则的计算网格,可以自适应地调整网格分辨率。这使得LDG法在处理复杂几何结构和材料界面时具有优势。因此,尽管已经有了许多研究DD模型和HF模型的方法,LDG法依然具有自己独特的优势。这正是研究LDG法相较于其他方法的创新点。