\documentclass[lang=cn,newtx,10pt,scheme=chinese]{elegantbook}

\setcounter{tocdepth}{2}

\newcommand\bs[1]{\boldsymbol{#1}}

\title{半离散局部不连续Galerkin方法在半导体器件仿真模型中的误差分析}

\begin{document}

\maketitle

\frontmatter

\tableofcontents

\mainmatter

\today
\section{介绍}
在我们之前的工作[1,2]中,我们已经开发了一种局部不连续Galerkin(LDG)有限元方法来解决半导体器件仿真中的时间依赖和稳态矩模型问题。这些模型中同时存在一阶导数对流项和二阶导数扩散(热传导)项,并且对流-扩散系统与泊松电势方程耦合。对流-扩散系统由局部不连续Galerkin(LDG)方法进行离散化[3, 4, 5],参见[6, 7, 8, 9]。
电场的电势方程也通过LDG方法进行离散化处理。通过使用LDG方法对一阶和高阶空间导数进行统一离散化,可完全发挥该方法在高度h-p自适应性和并行效率方面的潜力。
[1,2]中展示的数值结果表明该方法具有良好的分辨度,并与ENO有限差分方法[10]得到的结果达成一致。

局部不连续Galerkin(LDG)方法具有几个吸引人的特性[11]。
它们可以轻松地设计为任意阶的准确性。
实际上,准确度可以在每个单元中进行局部确定,从而实现高效的p自适应性。
它们可以用于任意的三角剖分,甚至包括具有悬挂节点的情况,从而实现高效的h自适应性。
该方法具有出色的并行效率。它们非常局部化:每个单元只需要与其直接相邻的单元进行通信,而与准确度无关。
此外,它们通常具有出色的可证明非线性稳定性。

在本文中,我们延续了我们在[1,2]中的工作,对具有平滑解的半导体器件的一维漂移-扩散(DD)模型和高场(HF)模型进行误差分析。我们注意到,在文献[12, 13]中,分别对解DD模型的P1连续有限元方法以及与P0-P1混合有限元方法耦合求解泊松方程的方法进行了分析。对于我们的情况,分析中的主要技术困难包括处理跨单元跳跃项,这些项在通过泊松求解器与非线性耦合时产生于我们的数值方法的不连续性。

对于求解线性守恒定律平滑解的DG方法,已在文献[14, 15, 16, 17]中给出了张量积网格的最优先验误差估计$O(h^{k+1})$,其他特殊网格的为O(h^{k+1}),其他情况下的为O(h^{k+1/2})。Cockburn和Shu在[3]中首次获得了线性对流-扩散方程的LDG方法的先验误差估计。之后,Castillo等人在[18, 19, 20]中证明了具有特定数值通量的LDG方法的最优收敛阶数为O(h^{k+1})。Rivi`ere和Wheeler[21]针对至少二次多项式的非线性对流-扩散方程给出了LDG方法的最优误差估计。最近,Zhang和Shu提出了具有平滑解的标量非线性守恒定律的全离散Runge-Kutta DG方法的先验误差估计[22],以及对称系统的先验误差估计[23]。Xu和Shu在[24]中为半离散局部不连续Galerkin方法应用于非线性对流-扩散方程和具有平滑解的KdV方程提供了L2误差估计。、

尽管LDG方法已经有了很多理论分析,但对于半导体器件力矩模型涉及到泊松势方程耦合的情况,这种分析似乎仍然不可用。本文给出了一维DD ( k≥1 )和HF ( k≥2 )模型LDG格式中使用$P^k$元素(k次分段多项式)时O( hk + 1②)的误差估计。本文还对这两个模型进行了仿真,以验证分析的正确性。
\end{document}