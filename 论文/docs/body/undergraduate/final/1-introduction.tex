\cleardoublepage

\section{绪论}

\subsection{引言}
半导体是一种电导率在绝缘体至导体之间的物质或材料。半导体器件利用此特性在电子技术、清洁能源、制造业和自动化等许多领域具有重要意义。半导体器件模拟则是现代半导体工业的重要研究课题之一。它不仅为半导体材料、微电子配件等相关领域的许多技术问题提供了理论上必要的解释,还能提供对器件行为和性能的预测和分析,利于半导体材料开发和优化。
我们将在本节分别论述:局部间断伽辽金法(LDG)的发展历史;龙格-库塔(RK)的发展历史;对半导体器件数学模型模拟的理论研究历史;最后给出本文的主要工作。
\subsection{局部间断伽辽金法的发展历史}
在讲解局部间断伽辽金法(LDG)法之前,我们首先介绍LDG法的前身间断伽辽金法(DGM)。1973年,Reed和Hill首次提出间断伽辽金法\cite{reed1973triangular}。
之后,Cockburn等发表了一系列论文极大发展了DGM,建立了DGM解决非线性时间依赖问题的基本框架\cite{reed1973triangular,cockburn1991runge,cockburn1989tvb2,cockburn1989tvb3,cockburn1990runge,cockburn1998runge},比如空气动力学的欧拉方程,使用显式非线性稳定的高阶龙格-库塔(RK)时间离散和DG空间离散化方法,通过精确或近似的Riemann求解器作为界面通量以及全变差有界(TVB)非线性限制器,在强激波条件下取得了非振荡性质。DGM在诸多领域中得到了迅速应用\cite{cockburn2000development}。2010年,Zhang和Shu给出了龙格-库塔间断伽辽金法(RKDG)在解决标量守恒定律的分析,其中采用三阶显式全变差不增龙格-库塔法进行时间离散,得到了对于一般数值通量的近似最优误差估计和迎风通量的最优误差估计\cite{zhang2010stabilitya}。

因为DGM解空间由在单元交界面上不连续的分段多项式组成,不足以处理高阶导数。这是有限元中的典型“不相容”情况。将DGM直接应用于包含二阶导数的热方程可能会得到计算中表现良好但实际上是错误的方法,并且对精确解存在$O(1)$的误差\cite{cockburn2001runge,zhang2003analysis}。对于包含高阶导数的时间相关偏微分方程,如对流-扩散系统耦合泊松电势方程,局部间断伽辽金(LDG)方法的想法是将方程重写为一阶系统,然后在该系统上应用DG方法。这个方法成功的关键在于正确设计数值通量。这些通量必须保证稳定性和所有引入的用于近似解的导数的辅助变量的局部可解性,。这些辅助变量局部可解性解释了为什么该方法被称为“局部”间断伽辽金法\cite{cockburn1998local}。DGM可以轻松地设计为任何精度。实际上,精度的阶数可以在每个单元中进行局部确定,这允许高效的p自适应性。它容易处理复杂几何和边界条件。它也可以用于任意三角剖分,甚至包括具有悬点的三角剖分,这样可以实现高效的h自适应性。由于它的数据交互具有明显的局部性,数值解的更新只需要相邻单元的信息,且不受精度影响,因此DGM可以实现高效的并行运算。此外,DGM也被证明具有良好的非线性稳定性。

1998年,Cockburn和Shu首次提出了解决具有二阶导的时间依赖对流-扩散系统的LDG法\cite{cockburn1998local}。他们的工作受到了Bassi和Rebay在可压缩Navier-Stokes方程中成功的数值实验的启发\cite{bassi1997high}。之后2002年,Yan和Shu提出了具有一般KdV类型方程的LDG法\cite{yan2002localb},同年他们给出了处理含有四阶和五阶空间导数的偏微分方程组的LDG方法\cite{yan2002local}。2002-2005年,Eskilsson和Sherwin提出了不连续谱元法来模拟一维线性博欣内斯克方程,离散浅水系统和二维博欣内斯克方程\cite{eskilsson2002discontinuous,eskilsson2005discontinuous,eskilsson2005discontinuousb}。2010年,Xu等提出解决高阶时间依赖PDE的LDG方法\cite{xu2010local},并证明了其好的性质。cockburn在2002年介绍了LDG法如何处理斯托克斯系统\cite{cockburn2002local}。2005年,Bustinza等给出了对于线性和分线性扩散问题在多边形区域内基于残差的可靠的后验误差估计\cite{bustinza2005posteriori}。2007年,cockburn提出解决对流-扩散问题和扩散问题的最小耗散LDG法\cite{cockburn2007analysis}。2009年,Burman和Stamm考虑混合形式LDG法的最小化稳定问题。
2015年,Wang和Shu给出了关于对流扩散问题LDG法耦合IMEX时间离散的误差估计\cite{wang2015stability}。

\subsection{龙格-库塔的发展历史}
许多具有强非线性稳定性的方案,比如全变差有界(TVD)和本质无震荡(ENO),需要结合保持解单调性的时间离散方案,如向前欧拉。但是由于欧拉法精度与高阶空间离散结合将会导致精度丢失,因此保持空间离散稳定性的高阶时间离散逐渐发展\cite{shu1988efficient},这种方案被称为TVDRK法,现在也叫做强保稳(SSP)龙格-库塔法。部分具有良好性质的三阶TVDRK法\cite{shu2007efficient}已经应用于了全离散LDG法并被分析\cite{wang2013errora}。这种时间离散在处理对流主导的对流扩散问题表现稳定,高效且准确,但对于非对流主导的对流扩散问题,显式时间离散对时间步长有严格的要求\cite{shu2007efficient}。由于对流扩散问题等偏微分方程含有不同类型的项,其中对流项通常是非线性的,我们更期望使用显式方法来解决非线性对流项,而采用隐式方法处理线性扩散项,因此自然我们考虑对两个项分别采用不同的处理方法,隐式-显式(IMEX)方法应运而生\cite{ascher1997implicitexplicita}。而如果对对流项和扩散项都使用隐式时间离散,得到的代数系统会不再椭圆,并且许多迭代求解器的收敛性将会受到影响。

现在IMEX格式已经被广泛应用于对流-扩散问题的时间离散,特别是于谱方法结合。1995年,Ascher等人对于原型线性对流-扩散方程,给出了一阶到四阶精度多步IMEX格式的稳定性分析,说明了稳定格式对于更广泛的问题可以允许更大的时间步长,证明了在扩散项强主导和一个合适的基于BDF格式被使用时,时间步长有着良好的限制\parencite{ascher1995implicit}。1997年,Ascher等人继续研究,给出了具有更好的稳定区间的IMEX RK法。2001年,Calvo构造了三阶和四阶的变步长线性隐式龙格-库塔法来解决对流-反应-扩散方程空间离散得到的半离散方程,并研究了这些方法的稳定性\cite{calvo2001linearly}。2003年,Kennedy等人研究additive龙格-库塔(ARK)法来解决空间离散的一维对流-扩散-反应方程,IMEX法可以看作ARK的特例。2004年,Pareschi和Russo对于含有刚性松弛系数的守恒双曲系统,采用SSP格式处理显式部分,采用L-稳定DIRK处理隐式部分得到了一种新的IMEX格式,这种格式被证明具有渐进保持(AP)性。
2013年,Wang和Zhang给出了解决一维Dirichlet边界条件线性对流扩散问题的全离散LDG算法,通过适当设置数值通量和中间边界条件,得到了时间和空间上的最优误差估计\cite{shu2007efficient}。
2015年,Wang和Shu等人分析了LDG法耦合特定IMEX时间离散得到的IMEX LDG格式的稳定性和误差估计,这种IMEX LDG格式对于线性对流-扩散问题具有无条件稳定性,也就是时间步长t只要小于某个常数,那么格式就是稳定的\cite{wang2015stability}。在第二节,我们会给出这篇文章中具体用到的IMEX格式,并且在数值算例部分我们的数值实验将验证其无条件稳定性。

\subsection{半导体器件模拟的理论研究历史}
传统的离散化方法,比如有限体积法\cite{bank1983numerical,bank1998finite,chainais2003finite,bessemoulin2012finite}和有限元法\cite{brezzi1989two,mauri20153d}过去被用于求解半导体模型。近年来,关于这类问题的各种数值研究在文献中不断涌现。其中局部间断伽辽金(LDG)凭借其高度的h-p自适应性、良好并行性、非线性稳定性等,越来越多地被应用到半导体器材模拟上。

2004年,Liu和Shu使用LDG方法处理流体动力模型(HD)和能量传输模型(ET)\cite{liu2004locala}。2007年,Liu和Shu进一步利用LDG法模拟了二维HD模型\cite{liu2007localb}。2007年,Chen等人研究了解决半导体器材的流体力学模型和HF模型LDG方法\cite{chen2007discontinuous}。2010年,Liu-Shu给出了处理一维DD和HF模型LDG法耦合TVDRK法格式\cite{liu2010error}。在DD模型和HF模型中,同时存在一阶导数对流项和二阶导数扩散(热传导)项,通过LDG方法对对流-扩散系统进行空间离散。但该方法仅对电子浓度方程进行离散化,对于电势方程,仍然使用连续方法,以避免在单元边界上出现两个独立解变量的不连续性。此外,由于电子浓度和电场的非线性耦合,当在LDG格式中使用$P^k$(分段多项式,次数为k)时,仅获得了次优误差估计$O(h^{k+\frac{1}{2}})$。2016年,Liu-Shu继续他们的研究,给出了半离散LDG格式与隐式-显式时间离散化LDG格式光滑解的最优误差估计\cite{liu2016analysis}。与之前的空间离散方法不同的是,在这篇文章中势能方程他们也通过LDG方法进行空间离散。通过使用LDG方法进行统一离散化,充分实现了LDG方法在简单h-p自适应性和并行效率方面的潜力。而IMEX格式相较于TVDRK格式大大放松了对时间步长的限制,节约了计算消耗。
2020年,Chen和Bagci使用Gummel方法对"解耦"和"线性化"泊松方程耦合静态漂移-扩散方程组成的系统,然后再用LDG方法离散得到的方程组\cite{chen2020steady}。通过与基于有限元和有限体积的模拟软件相比,证明了其准确性。2024年,Li等研究了一种弱伽辽金有限元方法,该方法利用k次分段多项式近似电子浓度和电势,同时用k+1次分段多项式来离散弱导数空间,最终得到了该方法再关于离散$H^1$范数和标准$L^2$范数的最优误差估计。

此外,差分法\cite{ding2019optimal},混合有限元\cite{gao2018linearized}和虚拟元法\cite{liu2021virtual}也被应用于解决半导体设备模拟模型。
\subsection{本文的主要工作}
本文主要研究半导体器件中两个重要的模型,它们分别是一维半导体器件的漂移扩散(DD)模型和高场(HF)模型。二者由泊松-玻尔兹曼方程推导得到,都含有一阶导对流项和二阶导扩散项,都可以通过耦合泊松方程可以用来计算半导体的电势分布和载流子的浓度分布。在本文的分析中,我们假设两个模型都具有光滑解。其中的主要难点在于如何处理数值解在单元边界的跳跃项,非线性和模型之间的耦合问题。第三节,我们将给出本文讨论的DD模型,并推导出它的TVD半离散LDG格式和IMEX全离散LDG格式,并给出其误差估计;第四节,我们将HF模型的方程,并推导出它的弱形式和半离散LDG格式;第五节,我们将给出具体的数值算例,探讨并论证基函数和初值函数的选择,同时证明部分误差估计的正确性;结论与展望将在第六节给出。
