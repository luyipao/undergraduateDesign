\cleardoublepage

\section{绪论}

\subsection{引言}
半导体是一种电导率在绝缘体至导体之间的物质或材料。半导体器件利用此特性在电子技术、清洁能源、制造业和自动化等许多领域具有重要意义。半导体器件模拟则是现代半导体工业的重要研究课题之一。它不仅为为半导体材料、微电子配件等相关领域的许多技术问题提供了理论上必要的解释,还能提供对器件行为和性能的预测和分析,利于半导体材料开发和优化。
我们将在本节分别论述:局部间断伽辽金法(LDG)的发展历史;龙格-库塔(RK)的发展历史;对半导体器件数学模型模拟的理论研究历史;最后给出本文的主要工作。
\subsection{局部间断伽辽金法的发展历史}
在讲LDG法之前,我们首先介绍LDG法的前身间断伽辽金法(DGM)。1973年,Reed和Hill首次提出DGM\cite{reed1973triangular}。
之后,Cockburn等发表了一系列论文极大发展了DGM,建立了DGM解决非线性时间依赖问题的基本框架\cite{reed1973triangular,cockburn1991runge,cockburn1989tvb2,cockburn1989tvb3,cockburn1990runge,cockburn1998runge},比如空气动力学的欧拉方程,使用显式非线性稳定的高阶Runge-Kutta(RK)时间离散和DG空间离散化方法,通过精确或近似的Riemann求解器作为界面通量以及total variation bounded(TVB)非线性限制器,取得了强激波下的非振荡性质。DGM在诸多领域中得到了迅速应用\cite{cockburn2000development}。

因为DGM解空间由在单元交界面上不连续的分段多项式组成,不足以处理高阶导数。这是有限元中的典型“不相容”情况。将DGM粗心地直接应用于包含二阶导数的热方程可能会得到计算中表现良好但与原方程“不一致”的方法,并且对精确解存在$O(1)$的误差\cite{cockburn2001runge,zhang2003analysis}。对于包含高阶导数的时间相关偏微分方程,如对流-扩散系统耦合Poisson电势方程,局部间断伽辽金(LDG)方法的想法是将方程重写为一阶系统,然后在该系统上应用DG方法。这个方法成功的关键在于正确设计数值通量。这些通量必须保证稳定性和所有引入的用于近似解的导数的辅助变量的局部可解性,。这些辅助变量局部可解性解释了为什么该方法被称为“局部”不连续Galerkin方法\cite{cockburn1998local}。

1998年,Cockburn和Shu首次提出了解决时间依赖对流-扩散系统的LDG法\cite{cockburn1998local}。他们的工作受到了Bassi和Rebay在可压缩Navier-Stokes方程中成功的数值实验的启发\cite{bassi1997high}。2010年,Xu等提出解决高阶时间依赖PDE的LDG方法\cite{xu2010local},并证明了其好的性质。它们可以轻松地设计为任何精度。实际上,精度的阶数可以在每个单元中进行局部确定,这允许高效的p自适应性。它们可以用于任意三角剖分,甚至包括具有悬点的三角剖分,这样可以实现高效的h自适应性。这些方法具有出色的并行效率,因为它们在本质上是非常局部的,每个单元只需要与其直接相邻的单元进行通信,而不受精度的影响。此外,这些方法具有优秀的可证明非线性稳定性。
\subsection{龙格-库塔的发展历史}
许多具有强非线性稳定性的方案,比如全变差有界(TVD)和本质无震荡(ENO),需要结合保持解单调性的时间离散方案,如向前欧拉。但是由于欧拉法精度与高阶空间离散结合将会导致精度丢失,因此保持空间离散稳定性的高阶时间离散逐渐发展\cite{shu1988efficient},这种方案被称为TVDRK法,现在也叫做强保稳(SSP)龙格-库塔法。部分具有良好性质的三阶TVDRK法\cite{shu2007efficient}已经应用于了全离散LDG法并被分析\cite{wang2013errora}。这种时间离散在处理对流主导的对流扩散问题表现稳定,高效且准确,但对于非对流主导的对流扩散问题,显式离散对时间步长有严格的要求\cite{shu2007efficient}。为了解决这个问题,一个直接的想法是使用隐式方法,但是对于对流扩散问题,对流项通常是非线性的,我们更期望使用显式方法来解决非线性对流项,而采用隐式方法处理线性扩散项。隐式-显式(IMEX)方法自然被得到\cite{ascher1997implicitexplicita}。而如果对对流项和扩散项都使用隐式时间离散,得到的代数系统会不再椭圆,并且许多迭代求解器的收敛性将会受到影响。

目前已经有多种IMEX格式被设计出来。其中,\parencite{ascher1995implicit}是多步格式,\parencite{ascher1997implicitexplicita}和更多的文献是基于RK法的IMEX方法。在\cite{cooper1983additive}中,小于等于四阶的一对半显性和线性方法被证明具有和被扰动的线性微分方程具有相似的稳定性。

\subsection{半导体器件模拟的理论研究历史}
传统的离散化方法,比如有限体积法\cite{bank1983numerical,bank1998finite,chainais2003finite,bessemoulin2012finite}和有限元法\cite{brezzi1989two,mauri20153d}过去被用于求解半导体模型。近年来,关于这类问题的各种数值研究在文献中不断涌现。其中局部间断伽辽金(LDG)凭借其高度的h-p自适应性、良好并行性、非线性稳定性等,越来越多地被应用到半导体器材模拟上。

2004年,Liu和Shu使用LDG方法处理流体动力模型(HD)和能量传输模型(ET)\cite{liu2004locala}。2007年,Liu和Shu进一步利用LDG法模拟了二维HD模型\cite{liu2007localb}。2007年,Chen等人研究了解决半导体器材的流体力学模型和HF模型LDG方法\cite{chen2007discontinuous}。2010年,Liu-Shu给出了一维DD和HF模型LDG方法\cite{liu2010error}。在该方法中,同时存在一阶导数对流项和二阶导数扩散(热传导)项,并且对流-扩散系统通过LDG方法进行离散化。该方法仅使用LDG方法对电子浓度方程进行离散化,对于电势方程,仍然使用连续方法,以避免在单元边界上出现两个独立解变量的不连续性。此外,由于电子浓度和电场的非线性耦合,当在LDG方案中使用$P^k$(分段多项式,次数为k)时,仅获得了次优误差估计$O(h^{k+\frac{1}{2}})$。2016年,Liu-Shu继续他们的研究,给出了半离散LDG格式与隐式-显式(IMEX)时间离散化LDG格式光滑解的误差估计\cite{liu2016analysis}。该方法相较于之前不同的是,势能方程也通过LDG方法进行离散化。通过使用LDG方法进行统一离散化,充分实现了该方法在简单h-p自适应性和并行效率方面的潜力。
2020年,Chen和Bagci推导出了一种"解耦"和"线性化"的方法来处理静态漂移-扩散方程,然后使用LDG格式离散化得到的方程\cite{chen2020steady}。2022年,Li等提出了一个新的DD模型有限体积方案,该方案还适用于退化情况并保持稳态\cite{li2022stabilized}。2024年,Li等推导了一维DD和HF模型的weak Galerkin(WG)。

此外,差分法\cite{ding2019optimal},混合有限元\cite{gao2018linearized}和虚拟元法\cite{liu2021virtual}也被应用于解决半导体设备模拟模型。
Wang和Shu分析了仔细挑选的IMEX RK时间离散LDG法的稳定性和误差估计\cite{wang2015stability,wang2016stability}。在相同的时间条件下,如果采用一般单调数值通量,针对阶IMEX RK格式和三阶多部IMEX格式耦合LDG空间离散,给出了空间和时间上的最优误差估计。
\subsection{本文的主要工作}
本文主要研究半导体器件中两个重要的模型,它们分别是一维半导体器件的漂移扩散(DD)模型和高场(HF)模型。二者由泊松-玻尔兹曼方程推导得到,都含有一阶导对流项和二阶导扩散项,都可以通过耦合泊松方程可以用来计算半导体的电势分布和载流子的浓度分布。在本文的分析中,我们假设两个模型都具有光滑解。其中的主要难点在于如何处理数值解在单元边界的跳跃项,非线性和模型之间的耦合问题。第三节,我们将给出本文讨论的DD模型,并推导出它的TVDLDG格式和IMEX全离散LDG格式,并给出其误差估计;第四节,我们将HF模型的方程,并推导出它的弱形式和半离散LDG格式;第五节,我们将给出具体的数值算例,探讨并论证基函数和初值函数的选择,同时证明部分误差估计的正确性;结论与展望将在第六节给出。
