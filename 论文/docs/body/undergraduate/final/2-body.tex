\cleardoublepage

\section{预备知识}
在本节,我们将给出一些符号和定义。在之后的部分,我们会直接使用这些符号和结论。
\subsection{基础符号}
首先定义计算域$[a,b]$的一个剖分
\begin{align*}
    I_{j} = (x_{j-1/2}, x_{j+1/2}), \ j = 1, 2, ..., N
\end{align*}
其中$a=x_{1/2} < x_{3/2}< ...< x_{N+1/2}=b$,N表示网格数。接着定义
\begin{align*}
    \Delta x_j = x_{j+1/2}-x_{j-1/2}, \quad h = \max{\sup_j{\Delta x_j}}, \quad x_j = (x_{j+1/2}+x_{j-1/2})/{2},
\end{align*}
其中$\Delta x_j$表示网格大小,$h$表示网格长度,$x_j$表示网格中点。

网格上有限维计算空间$V_h^k = \{v:v|_{I_j}\in P^k(I_j); 1\leq j\leq N\}$,其中$P^k(I_j)$表示$I_j$上次数不大于$k$的多项式集合。我们的数值解和测试函数都将从$V_h^k$中取得。需要注意的是,$V_h^k$中的函数在单元边界节点处不一定是连续的,可以出现跳跃。

为了简化标记,我们分别定义$(u_h)^+_{j+1/2}=u_h(x^+_{j+1/2})$和$(u_h)^-_{j+1/2}=u_h(x^-_{j+1/2})$。此外我们定义$u_h$在单元边界节点$x_{j+1/2}$的跳跃$[u_h]_{j+1/2}=(u_h)^+_{j+1/2}-(u_h)^-_{j+1/2}$和均值$(\overline{u}_h)_{j+1/2}=((u_h)_{j+1/2}^++(u_h)_{j+1/2}^-)/2$。
\subsection{初值的处理}
我们的初值函数就是定义在$[0, 0.6]$的掺杂函数$n_d$,它满足
\begin{align*}
    n_d(x) = \begin{cases}
                 5\times 10^{17}, \quad x \in [0, 0.1] \cup [0.5, 0.6] \\
                 2\times 10^{15}, \quad x \in [0.15,0.45]
             \end{cases}
\end{align*}
在中间过渡区域选择光滑的函数连接。在本文中我们考虑光滑过渡函数$g(x)$,满足
\begin{align*}
    f(x)               & ={\begin{cases}
                               e^{-{\frac {1}{x}}}, \quad x>0 \\
                               0, \quad x\leq 0\end{cases}} ,                     \\
    \displaystyle g(x) & :={\frac {f(x)}{f(x)+f(1-x)}}\quad x\in \mathbb {R}.
\end{align*}
为了得到区间$[a,b]$上的光滑过渡,我们只需要考虑函数
\begin{align*}
    g(\frac{x-a}{b-a}).
\end{align*}
注意到它任意阶导函数在$[a,b]$外都为零,结合掺杂函数我们容易得到一个任意阶连续导函数的初值函数。这不是唯一可选择的过渡函数,但由于初值函数的光滑性与计算空间的维数有关,选择这样一个绝对光滑的初值函数可以简化后续分析。

为了让初值函数落在我们的计算空间中,我们需要对其进行处理。本文选择的是向计算空间$V_h^k$的分段$L^2$投影,表示为$\mathcal{P}$。对于任意函数$u\in C^{k+1}$,它满足
\begin{align}
    \int_{I_j}(\mathcal{P}u(x)-u(x)v(x)) \rm{d}x = 0, \quad \forall v \in P^k(I_j).
\end{align}

\section{DD模型}
本文采用的DD模型表示为
\begin{align}
    n_t - (\mu En)_x = \tau \theta n_{xx}, \label{eq:DD} \\
    \phi_{xx} = \frac{e}{\epsilon}(n - n_d),  \label{eq:poisson},
\end{align}
其中$x\in(0,1)$,第一个电子浓度方程取周期边界条件,第二个电势方程取Dirichlet边界条件:$\phi(0,t) = 0, \phi(1,t) = v_{bias}$。
系统\ref{eq:DD}-\eqref{eq:poisson},未知量是电子浓度$n$和电势$\phi$。$m_0$表示电子有效质量,k是Boltzmann常数,e是电子电荷,$\mu$代表迁移率,$T_0$是晶格温度,$\tau = \frac{m_0 \mu}{e}$是松弛参数,$\theta = \frac{k}{m_0}T_0$,$\epsilon$是介电常量,$n_d$是一个给定的掺杂函数。

\subsection{弱形式和LDG格式}

% \sectionnonum[none]{同一页上的章标题}
