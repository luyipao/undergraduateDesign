\cleardoublepage{}
\begin{center}
    \bfseries \zihao{3} 摘~要
\end{center}
本文考虑半导体模型中具有光滑解的一维漂移-扩散(DD)模型和一维高场(HF)模型,其中包含一阶导非线性对流项和二阶导线性扩散项,并耦合泊松方程。文中给出了两个模型在局部间断伽辽金(LDG)空间离散耦合全变差不增龙格-库塔法时间离散下的误差估计。此外对于DD模型,文中也给出了它关于隐式-显式(IMEX)龙格-库塔(RK)法时间离散的误差估计。解决这两个模型的主要难点在于数值解在单元边界处的不连续性,因此需要选择合适的数值通量来处理单元边界处的跳跃。IMEX法的想法是采用隐式方法处理非线性对流项,采用显式方法处理扩散项。这样做在保证解无条件稳定性的同时减小时间步长,降低计算消耗。最后给出数值算例来验证文中结论。

\noindent \textbf{关键词}:局部间断伽辽金法,半离散,隐式-显式格式,误差估计,半导体
\cleardoublepage{}
\begin{center}
    \bfseries \zihao{3} Abstract
\end{center}
We consider the one-dimensional drift-diffusion (DD) model and the one-dimensional high-field (HF) model with smooth solutions in the semiconductor model, which include the first-order derivative nonlinear convection term and the second-order derivative linear diffusion term, and are coupled with the Poisson's equation. The main difficulty in solving both models includes the inter-element jump caused by the discontinuous nature of local discontinuous Galerkin (LDG) method, so it is necessary to choose an appropriate numerical flux to handle the jump at the boundary of the element. Error estimates for both models are given under LDG spatial discretization with total variation diminishing (TVD) Runge-Kutta (RK) for time discretization. In addition, for the DD model, an error estimate is also given for its implicit-explicit (IMEX) RK time discretization. The idea of the IMEX method is to treat the nonlinear convection term with an implicit method and treat the diffusion term with an explicit method. It reduces the time step size, thus reducing computational consumption, while ensuring the unconditional stability of the solution. Finally, numerical examples are provided to verify the conclusions.

\noindent \textbf{Keywords}: Local discontinuous Galerkin method, semi-discrete, implicit-explicit scheme, error estimate, semi-conductor