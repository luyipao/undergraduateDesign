\cleardoublepage
\newrefsection
\chapter{文献综述}

\section{背景介绍}
半导体是一种电导率在绝缘体至导体之间的物质或材料。半导体器件利用此特性在电子技术、清洁能源、制造业和自动化等许多领域具有重要意义。半导体宏观数学模型则是是现代半导体工业的重要研究课题之一。它不仅为半导体材料、微电子配件等相关领域的许多技术问题提供了理论上必要的解释,还提供对器件行为和性能的预测和分析,利于半导体材料开发和优化。
\subsection{DD模型\cite{cercignani2000device}和HF模型\cite{cercignani2000device}}
漂移-扩散(Drift-Diffusion,简称DD)模型是一种相对简单且使用广泛的半导体材料宏观数学模型。该模型用于描述半导体器件中的载流子输运行为,特别是针对低频和直流条件下的行为。高场模型(High-field model,简称HF)是半导体宏观数学模型的一种,用于描述在高电场条件下的半导体器件行为。二者来源于描述半导体器件中电子输运的经典Boltzmann-Poisson系统。这两个模型都存在一阶导数对流项和二阶导数扩散(热传导)项,并且对流-扩散系统与Poisson电势方程耦合\cite{cercignani2000device}。分析中的主要技术难点是处理电子势和电子浓度之间的非线性耦合。此外,在半导体器件中存在尖锐的不连续掺杂剖面,这对数值模拟构成了另一个挑战。

\subsection{discontinuous Galerkin (DG)方法}
1973年,Reed和Hill首次提出DG法\cite{reed1973triangular}。
Cockburn等发表了一系列论文极大发展了DG法,建立了DG法解决非线性时间依赖问题的基本框架\cite{reed1973triangular,cockburn1991runge,cockburn1989tvb2,cockburn1989tvb3,cockburn1990runge,cockburn1998runge},比如空气动力学的欧拉方程,使用显式非线性稳定的高阶Runge-Kutta(RK)时间离散和DG空间离散化方法,通过精确或近似的Riemann求解器作为界面通量以及total variation bounded(TVB)非线性限制器,取得了强激波下的非振荡性质。

DG方法在诸多领域中得到了迅速应用,包括声学、电磁学、气体动力学、颗粒流动、磁流体力学、气象学、浅水模拟、海洋学、油藏模拟、半导体器件模拟、多孔介质中的污染物传输、涡轮机械、湍流流动、粘弹性流体和天气预报等\cite{cockburn2000development}。

\subsubsection{local discontinuous Galerkin (LDG)方法}
对于包含高阶空间导数的方程,如对流-扩散方程
\begin{equation}\label{pde:convection-diffusion}
    u_t + \sum f_i(u)_{x_i} - \sum \sum (a_{i,j}(u)u_{x_j})_{x_i} = 0,
\end{equation}
DG方法不能直接应用。这是因为解空间由在单元交界面上不连续的分段多项式组成,不足以处理高阶导数。这是有限元中的典型“不相容”情况。将DG方法粗心地直接应用于包含二阶导数的热方程可能会得到计算中表现良好但与原方程“不一致”的方法,并且对精确解存在O(1)的误差\cite{cockburn2001runge,zhang2003analysis}。

对于包含高阶导数的时间依赖偏微分方程,如对流-扩散方程$\eqref{pde:convection-diffusion}$,LDG方法的思想是将方程重写为一阶系统,然后在该系统上应用DG方法。这个方法成功的关键因素是正确设计界面数值通量。这些通量必须保证稳定性和所有引入的辅助变量的局部可解性,这些辅助变量用于逼近解的导数。所有辅助变量的局部可解性解释了为什么该方法被称为“局部”不连续Galerkin方法\cite{cockburn1998local}。
1998年,Cockburn和Shu首次提出了解决时间依赖对流-扩散系统$\eqref{pde:convection-diffusion}$的LDG法\cite{cockburn1998local}。他们的工作受到了Bassi和Rebay在可压缩Navier-Stokes方程中成功的数值实验的启发\cite{bassi1997high}。

2010年,Xu等提出解决高阶时间依赖PDE的LDG方法\cite{xu2010local},并证明了其好的性质。它们可以轻松地设计为任何精度。实际上,精度的阶数可以在每个单元中进行局部确定,这允许高效的p自适应性。它们可以用于任意三角剖分,甚至包括具有悬点的三角剖分,这样可以实现高效的h自适应性。这些方法具有出色的并行效率,因为它们在本质上是非常局部的,每个单元只需要与其直接相邻的单元进行通信,而不受精度的影响。此外,这些方法具有优秀的可证明非线性稳定性。

\section{国内外研究现状}
对于半导体器件模拟的理论和数值研究具有悠久的历史。传统的离散化方法,如有限体积法\cite{bank1983numerical,bank1998finite,chainais2003finite,bessemoulin2012finite}和有限元法\cite{brezzi1989two,mauri20153d}被应用于求解半导体模型。近年来,关于这类问题的各种数值研究在文献中不断涌现。其中局部间断伽辽金法(LDG)凭借其高度的h-p自适应性、良好并行性、非线性稳定性等,越来越多地被应用到半导体器材模拟上。
\subsection{研究方向及进展}

2010年,Liu-Shu给出了一维DD和HF模型LDG方法\cite{liu2010error}。在该方法中,同时存在一阶导数对流项和二阶导数扩散(热传导)项,并且对流-扩散系统通过LDG方法进行离散化。该方法仅使用LDG方法对电子浓度方程进行离散化,对于电势方程,仍然使用连续方法,以避免在单元边界上出现两个独立解变量的不连续性。此外,由于电子浓度和电场的非线性耦合,当在LDG方案中使用$P^k$(分段多项式,次数为k)时,他们仅获得了次优误差估计$O(h^{k+\frac{1}{2}})$。
2016年,Liu-Shu继续他们的研究,给出了半离散LDG格式与隐式-显式(IMEX)时间离散化LDG格式光滑解的误差估计\cite{liu2016analysis}。该方法相较于之前不同的是,势能方程也通过LDG方法进行离散化。通过使用LDG方法进行统一离散化,充分实现了该方法在简单h-p自适应性和并行效率方面的潜力。
2020年,Chen和Bagci推导出了一种"解耦"和"线性化"的方法来处理静态漂移-扩散方程,然后使用LDG方法离散化得到的方程\cite{chen2020steady}。
2007年,Chen等人研究了解决半导体器材的流体力学模型和HF模型LDG方法\cite{chen2007discontinuous}。
2022年,Li等提出了一个新的DD模型有限体积方案,该方案还适用于退化情况并保持稳态\cite{li2022stabilized}。
此外,差分法\cite{ding2019optimal},混合有限元\cite{gao2018linearized}和虚拟元法\cite{liu2021virtual}也被应用于解决半导体设备模拟模型。
Wang和Shu分析了仔细挑选的IMEX RK时间离散耦合LDG法的稳定性和误差估计\cite{wang2015stability,wang2016stability}。在相同的时间条件下,如果采用一般单调数值通量,针对阶IMEX RK格式和三阶多部IMEX格式耦合LDG空间离散,给出了空间和时间上的最优误差估计。
这也是本文中采用的IMEX RK时间离散方法。其分析推广到多维非线性对流-扩散问题需要一些技术处理,以获得数值解及其梯度之间的关键关系。
同时,不能直接将误差估计从一维推广到多维,特别是对于一般的三角形网格。
2024年,Li等推导了一维DD和HF模型的弱伽辽金法\cite{li2024weak}。

\subsection{存在问题}
\subsubsection{物理模型}
目前的LDG方法应用的物理模型比较简单,对于非线性,高维问题需要更进一步的分析。
\subsubsection{界面条件}
在LDG法中,界面条件的处理是一个重要的问题。特别是当界面条件不连续或具有跳跃项时,需要选取合适得数值通量来确保界面条件的准确性和稳定性。目前还没有明确的数学方法来选择合适的数值通量,更多地依赖物理背景来推测大概适用的数值通量。
\subsubsection{网格依赖性}
LDG法的解对网格的分辨率和形状敏感。在不适当的网格上,可能会出现数值振荡或收敛性差的情况。Wang和Shu对于几种特定IMEX RK时间推进LDG法的稳定性和误差估计针对的是均匀网格,对于非均匀及更复杂的网格,还缺少完善的理论分析。
\subsubsection{高阶离散化方法}
尽管LDG法可以轻松扩展到高阶离散化方法,但目前的研究主要集中在三阶及以下的离散。高阶离散化可能导致更复杂的数值通量计算和更高的计算成本。因此,为了平衡高阶离散化和计算效率之间的权衡,通常选择舍弃更高阶的算法。
\subsubsection{多物理耦合}
在处理半导体器件模型时,通常需要考虑多个物理过程的耦合,如电磁场和热传导的耦合。在LDG法中实现多物理耦合可能需要进一步的研究和开发。

\section{研究展望}
\subsection{物理模型}
目前的LDG方法主要应用于低维DD模型和HF模型,未来的可以考虑处理高维DD模型和HF模型,半导体设备模型领域更复杂的物理模型,如能带模型(Band Structure Model)、量子输运模型(Quantum Transport Model)等等。更近一步,未来也可以探究当前的LDG方法在非半导体物理模型效果。
\subsection{高阶离散化方法}
利用LDG方法的灵活性和高自由度,可以方便地扩展到高阶离散化方法。正如前文已经提到了LDG法耦合三阶IMEX RK格式和三阶多步IMEX格式\cite{wang2016stability}。未来可以进一步探索基于LDG的高阶方法,以提高数值解的精度和收敛速度。
\subsection{多物理耦合}
半导体器件模拟通常涉及多个物理过程的耦合,如电场、热传导、电磁辐射等。半导体设备矩模型耦合泊松势能方程的LDG方法理论分析并不可用\cite{liu2010error}。可见两个及多个物理模型耦合问题的理论研究不容易推广到其他模型,因此对于每个具体模型都要重新分析。
\subsection{并行计算}
通常来讲,为了提高计算效率,我们一般考虑三阶内的时间离散方案\cite{wang2016stability}。通过研究LDG方法的并行算法和优化策略提高计算速度,不仅可以加快计算效率,也为探索更高阶离散化方法提供了可能。

\subsection{不规则网格}
在实际应用中,半导体器件的几何形状往往是复杂的,因此使用不规则网格进行离散化是必要的。未来的研究可以关注LDG方法在不规则网格上的适应性和高效性。此外由于实际问题中变量(如浓度)变化存在急缓,因此对于变换迅速的区间采用更密集网格允许我们采用更长的时间步长,进而节省计算资源,提高计算效率\cite{肖红单2023半导体}。


\newpage
\begingroup
\linespreadsingle{}
\printbibliography[title={参考文献}]
\endgroup
